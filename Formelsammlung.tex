% This File is (c) by Dirk Winkel
% Licenced under GPL v3 or newer

\documentclass[12pt,a4paper,oneside]{article}
% -----------------------------------------------------------------------------------------
\usepackage[utf8]{inputenc}
\usepackage[centertags]{amsmath}
\usepackage{amsfonts}
\usepackage{amssymb}
\usepackage{amsthm}
\usepackage{newlfont}
\usepackage{amsxtra}
\usepackage{amstext}
\usepackage{latexsym}
\usepackage[ngerman]{babel}
\usepackage[top=2cm,bottom=2cm,left=2cm,right=2cm]{geometry}
\usepackage{pifont}
\usepackage{ulem}
\usepackage{color}
\usepackage{marvosym}
\usepackage[T1]{fontenc}
\usepackage{amsfonts}
\usepackage{graphicx}
\usepackage{array}
\usepackage{booktabs}
\usepackage{dcolumn}
\usepackage{units}
\usepackage{rotating}
\usepackage{hyperref}
\usepackage{floatflt}
%\usepackage{makeidx}
\usepackage{fancyhdr}
\usepackage{totpages}
\usepackage{amsmath}
\usepackage{framed}
\usepackage{multicol}
\usepackage{lmodern}

\usepackage{mathtools}
% -----------------------------------------------------------------------------------------
\begin{document}

-----------------------------------------------------------------------------------------
%	 Titel
% -----------------------------------------------------------------------------------------
%\headheight0cm\headsep0cm\topskip0cm\footskip0cm
\pagestyle{fancy} %eigener Seitenstil
\fancyhf{} %alle Kopf- und Fußzeilenfelder bereinigen
\fancyhead[L]{Gymnasiale Oberstufe} %Kopfzeile links
\fancyhead[C]{Kurz-Formelsammlung Physik} %zentrierte Kopfzeile
\fancyhead[R]{Seite \thepage/\ref{TotPages}} %Kopfzeile rechts
\renewcommand{\headrulewidth}{0.1mm} %obere Trennlinie
\headheight1cm

\textheight25cm
\setlength{\parindent}{0em}
%\fancyfoot[C]{\thepage} %Seitennummer
%\renewcommand{\footrulewidth}{0.4pt} %untere Trennlinie

\DeclarePairedDelimiter{\abs}{\lvert}{\rvert}

\setlength{\abovedisplayskip}{7pt}
\setlength{\belowdisplayskip}{7pt}
~
\begin{center}
\vspace{-1.8cm}
{\Large \textbf{Physikalische Formeln für die Oberstufe}}
\end{center}
% -----------------------------------------------------------------------------------------
%	Hauptext
% -----------------------------------------------------------------------------------------
\setlength{\columnseprule}{0.1mm}
\begin{multicols}{3}

\subsection*{Mechanik}
Gleichförmige Bewegung:
$$ \vec{s}=\vec{v} \cdot t $$
gleichmäßig beschleunigte Bewegung:
$$ \vec{s}=\frac{1}{2}\vec{a}t^2+\vec{v_0}t+\vec{s_0} $$
$$ \vec{v}=\vec{a} \cdot t $$
Kraft und Beschleunigung:
$$ \vec{F}=m \cdot \vec{a} $$
Feder/Hooke'sches Gesetz:
$$ \vec{F}=-D \cdot \vec{s} $$
Kreisbewegung:
$$ \vec{F}_z=\frac{m \cdot \vec{v}^2}{\abs{\vec{r}}} $$
Impuls:
$$ \vec{p} = m \cdot \vec{v} $$
Energie:
$$ E_{kin}=\frac{1}{2}m\vec{v}^2 $$
$$ E_{pot}=m \cdot g \cdot h $$
$$ E_{Feder}=\frac{1}{2}D \cdot \vec{s}^2 $$
Arbeit (mechanisch):
$$ W=\Delta E=\vec{F} \cdot \vec{s} $$
Gravitationsgesetz:
$$ \abs{\vec{F}}=G\frac{m_1 m_2}{\abs{\vec{r}}^2} $$

\subsection*{elektrische- und\\ magnetische Felder}
Coulombkraft:
$$ \abs{\vec{F}}=\frac{1}{4\pi \varepsilon_0}\frac{q \cdot Q}{\abs{\vec{r}}^2} $$
El. Feldstärke:
$$ \vec{E}=\frac{\vec{F}}{q} $$
El. Potentialdifferenz:
$$ \frac{\Delta E_{pot}}{q} = \Delta \Phi =  U $$
Plattenkondensator:
$$ \vec{E}=\frac{U}{\vec{d}} $$
$$ C=\varepsilon_0\varepsilon_r\frac{A}{d} = \frac{Q}{U} $$
$$ E = \frac{1}{2} C \cdot U^2 $$
Lorenzkraft:
$$ \vec{F_L}=q\cdot \vec{v} \times \vec{B} $$
Induktionsspannung:
$$ U_{ind}=\abs{\vec{B}\cdot l \times \vec{v}} $$
Induktivität (Spule):
$$ L = \mu_0 \mu_r \cdot \frac{n^2}{l}A $$
Magn. Feld (lange Spule):
$$ \abs{\vec{B}}=\mu_0 \mu_r \frac{n\cdot I}{l} $$
Energie (lange Spule):
$$ E = \frac{1}{2}L\cdot I^ 2 $$

\subsection*{Schwingungen\\ und Wellen}
Allgemeine Schwingungsgleichung:
$$ A \cdot sin(\omega t - \varphi) $$
Schwingungsdauer und\\
(Kreis-)Frequenz:
$$ \omega = 2\pi f = \frac{2\pi}{T} $$
Federpendel:
$$ T = 2\pi \sqrt{\frac{m}{D}} $$
Fadenpendel:
$$ T = 2\pi \sqrt{\frac{l}{g}} $$
El. Schwingkreis:
$$ T = 2\pi \sqrt{L \cdot C} $$
Wellenlänge:
$$ \lambda = c \cdot T = \frac{c}{f} $$
akkustischer Dopplereffekt:\\
($E$-Position positiv/``rechts'')
$$ f' = f \cdot \frac{c-v_E}{c-v_S} $$
optischer Dopplereffekt:
$$ f = f' \cdot \sqrt{\frac{c-v}{c+v}} $$

\subsection*{Quantenphysik}
Interferenzbedingung:
$$ g \cdot sin(\alpha) = n \cdot \lambda $$
Bragg-Bedingung:
$$ 2 g \cdot sin(\alpha) = n \cdot \lambda $$
deBroglie-Wellenlänge:
$$ \lambda = \frac{h}{p} $$
Photonenenergie:
$$ E = h \cdot f $$
Heisenberg'sche Unschärfe:
$$ \Delta E \cdot \Delta t = \Delta x \cdot \Delta p \geq \frac{h}{2\pi} = \hbar $$

\subsection*{Relativitätstheorie}
Energie-Masse-Beziehung:
$$ E = mc^2 $$
Lorentz-Faktor:
$$ \gamma = \sqrt{1-(\frac{v}{c})^2} $$
Lorenz-Transformation:\\
(analog für: $s, l, t'$)
$$ m = m_0 \gamma $$

\subsection*{weitere}
Zerfallsgesetz:
$$ N = N_0 \cdot 2^{-\dfrac{t}{t_{\frac{1}{2}}}} $$
Universelle Gasgleichung:
$$ p \cdot V = n_{Teilchen} \cdot k_B \cdot T $$


\end{multicols}
\emph{Hinweis:} Formelzeichen mit Pfeien ($\vec{Z}$) sind richtungsabhängige (=\textit{vektorielle}) Größen, ggf. muss die Richtung, z.B. als Vorzeichen, beachtet werden!

Ist dabei das \textit{Kreuzprodukt} $\times$ angegeben müsssen die Größen senkrecht zueinander sein um mit den Zahlenwerten (= \textit{Beträgen} $\abs{\vec{Z}}$) und $\cdot$ rechnen zu können. ($\rightarrow$ linke/rechte-Hand-Regeln)

%\newpage
\subsection*{Formelzeichen und (Grund-)Einheiten}
\begin{multicols}{3}
 \subsubsection*{Mechanik}
 $\vec{s}$: Strecke ($m$) \\
 $\vec{r}/\vec{d}$: Radius/Abstand ($m$)\\
 $\vec{v}$: Geschwindigkeit ($\frac{m}{s}$) \\
 $\vec{a}$: Beschleunigung ($\frac{m}{s^2}$) \\
 $t$: Zeit ($s$) \\
 $\vec{F}$: Kraft ($N=\frac{kg \cdot m}{s^2}$) \\
 $\vec{p}$: Impuls ($\frac{kg \cdot m}{s}$) \\
 $D$: Federkonstante ($\frac{N}{m}$) \\
 $E$: Energie ($J=Nm$) \\
 $p$: Druck ($p=\frac{N}{m^2}$) \\
 $W$: Arbeit=Energieänd. ($J$)
 \subsubsection*{El. \& Magn. Felder}
 $q$: Ladung (einzeln) ($C=As$) \\
 $Q$: Gesamtladung ($C$) \\
 $U$: Spannung ($V=\frac{J}{C}$) \\
 $A$: Fläche ($m^2$) \\
 $C$: Kapazität ($F=\frac{C}{V}$) \\
 $\vec{B}$: Magn. Flussdichte ($T$) \\
 $\vec{E}$: el. Feldstärke ($\frac{N}{C}$) \\
 $R$: el. Widerstand ($\Omega=\frac{V}{A}$) \\
 $L$: Induktivität ($H = \Omega s$) \\
 \subsubsection*{Schwingungen}
 $\omega$: Kreisfrequenz ($Hz = \frac{1}{s}$) \\
 $T$: Schwingungsdauer ($s$) \\
 $f$:Frequenz ($Hz$) \\
 $\lambda$: Wellenlänge ($m$) \\
 \subsubsection*{weitere}
 $n$: \textit{ganze Zahl}\\
 $\Delta$: \textit{Differenz/Änderung}\\
\end{multicols}


\subsection*{Wichtige Konstanten}
\begin{multicols}{3}
 Erdbeschl.(Mitteleuropa): $$ 9,81 \frac{m}{s^2} $$ \\
 Erdmasse: $$ 5,974 \cdot 10^{24} kg $$ \\
 Erdradius (mittlerer): $$ 6367 km $$ \\
 Schallgeschwindigkeit (ca): $$ c \approx 340\frac{m}{s} $$ \\
 Lichtgeschwindigkeit: $$ 2,998*10^8 \frac{m}{s} $$ \\
 Gravitationskonstante: $$ 6,674\cdot 10^{-11}\frac{m^3}{kg \cdot s^2} $$\\
 Elektrische Feldkonstante: $$ \varepsilon_0 = 8,854\cdot 10^{-12}\frac{As}{Vm} $$ \\
 Magnetische Feldkonstante: $$ \mu_0 = 4 \pi \cdot 10^{-7}\frac{N}{A^2} $$ \\
 Elementarladung: $$ e = 1,602\cdot 10^{-19} C $$ \\
 Elektronenvolt: $$ 1eV = 1,602\cdot 10^{-19}J $$ \\
 Elektronenmasse: $$ m_e = 9,109\cdot 10^{-31} kg $$ \\
 Protonenmasse: $$ m_p = 1,673 \cdot 10^{-27} kg $$ \\
 Planck'sches Wirkungsquantum: $$ h = 2\pi\hbar = 6,626\cdot 10^{-34}Js $$ \\
 Avogadro-Konstante: $$ 6,022\cdot 10^{23} \frac{1}{mol}$$ \\
 Boltzmann-Konstante: $$ k_B = 1,381\cdot 10^{-23} \frac{J}{K} $$
\end{multicols}

\cfoot{\tiny{Version 1.8.0 -- \copyright 2014-2019 Dirk Winkel, \LaTeX-Source and document licenced under GPL version 3 or newer -- download: polarwinkel.de}}

\end{document}
